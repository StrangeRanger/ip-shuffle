In the realm of network security, the concept of Moving Target (MT) techniques 
has gained significant traction. These techniques operate across various facets 
of computer systems with the aim of altering elements susceptible to exploitation 
by potential attackers. When considering the application of MT techniques within 
network defense strategies, dynamic IP address assignment, as employed in IP 
shuffling, emerges as a prominent example. By dynamically changing IP addresses,
network interfaces can effectively obscure targets from potential attackers, 
making it more challenging for them to identify and exploit vulnerabilities. 
This proactive approach aligns with the broader goals of MT techniques, which 
prioritize enhancing system resilience against cyber threats. Notably, while 
certain MT techniques like Address Space Layout Randomization (ASLR) have achieved
widespread adoption in modern operating systems, the implementation of IP shuffling 
represents an additional layer of defense that can complement existing security measures.
Through the adoption of IP shuffling and other MT techniques, organizations can 
strengthen their overall security posture and mitigate the impact of cyber threats. 
For instance the practice of dynamically changing IP addresses, randomizing memory 
layouts, and employing temporary encryption for memory contents fall within the 
spectrum of MT techniques ~\cite{okhravi2013finding}.
