In the realm of network security, Moving Target (MT) techniques have gained significant traction. These techniques operate across various facets of computer systems, aiming to alter elements susceptible to exploitation by potential attackers. Dynamic IP address assignment, as employed in IP shuffling, emerges as a prominent example when considering the application of MT techniques in network defense strategies. By dynamically changing IP addresses, network interfaces can obscure targets from potential attackers, making it more challenging for them to identify and exploit vulnerabilities.
This proactive approach aligns with the broader goals of MT techniques, which prioritize enhancing system resilience against cyber threats. While certain MT techniques like Address Space Layout Randomization (ASLR) have achieved widespread adoption in modern operating systems, implementing IP shuffling provides an additional layer of defense that can complement existing security measures. By adopting IP shuffling and other MT techniques, organizations can strengthen their overall security posture and mitigate the impact of cyber threats.
For instance, dynamically changing IP addresses, randomizing memory layouts, and employing temporary encryption for memory contents fall within the spectrum of MT techniques~\cite{okhravi2013finding}.