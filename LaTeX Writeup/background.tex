Our threat model concerns the scenario in which a system is attacked. Specifically, we focus on the scenario depicted in the diagram, where 
three interconnected computers form a network, with one of these computers compromised. Within this context, our threat model revolves around 
an attacker who has successfully gained access to one of the systems, as illustrated in the diagram. Once inside the network, the attacker's 
assumed objective is to scan other systems to identify vulnerabilities for lateral movement. The provided script, named "ip-shuffle," plays a crucial 
role in this threat scenario, as it allows for the dynamic assignment of random IP addresses to network interfaces. The attackers capabilities 
could possibly be but not limited to exploitation skills because of the knowledge of common vulnerabilities, network reconnaissance
by scanning the network to identify other vulernabilities, and persistence. The "ip-shuffle" with the random timing and frequency of IP address changes
makes it harder for attackers to predict when the changes will occur for the attackers assumed abilites. It also makes it difficult to map out the 
network accurately, this can confuse attackers and make it harder for them to identify and target vulnerable systems.

This proactive approach aligns with the broader goals of MT techniques, which prioritize enhancing system resilience against cyber 
threats. While certain MT techniques like Address Space Layout Randomization (ASLR) have achieved widespread adoption in modern operating 
systems, implementing IP shuffling provides an additional layer of defense that can complement existing security measures. By adopting IP 
shuffling and other MT techniques, organizations can strengthen their overall security posture and mitigate the impact of cyber threats.
For instance, dynamically changing IP addresses, randomizing memory layouts, and employing temporary encryption for memory contents fall 
within the spectrum of MT techniques~\cite{okhravi2013finding}.