The Moving Target Defense (MTD) technique we're working towards is IP shuffling, aimed at complicating lateral movement reconnaissance. This strategy involves dynamically changing the IP addresses of systems on a network. In our model, we have a private subnet containing three virtual machines that perform IP address rotation, periodically or erratically shifting across 254 different IP addresses.
Our diagram illustrates a scenario where one of these machines, denoted as Computer 2, has been compromised. By continuously changing IP addresses in an unpredictable manner, IP shuffling impedes attackers' reconnaissance efforts, making it difficult for them to identify and exploit vulnerabilities. The diagram delineates the intricate architecture of our network infrastructure, illustrating the hierarchical arrangement of networks, subnets, and their corresponding topological relationships. Within this schematic representation, the compromised computer is depicted, providing a visual reference to its position within the broader network.