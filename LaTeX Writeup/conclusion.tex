Moving Target Defense (MTD) has been hailed as a revolutionary strategy in cybersecurity that increases complexity and costs for attackers while reducing the exposure of vulnerabilities and enhancing system resilience \cite{cai2016network}. This paper introduced the \texttt{ip-shuffle} script, a robust solution for dynamically allocating random IP addresses to network interfaces, thereby impeding attackers' reconnaissance efforts.
The \texttt{ip-shuffle} script provides a systematic approach to dynamic IP address assignment through its modular design and comprehensive functionalities, including generating random IP addresses, verifying availability, and validating network configurations. By leveraging error-handling mechanisms and Unix signal responsiveness, the script ensures reliable execution and strengthens network resilience. The evaluation demonstrated the impact of \texttt{ip-shuffle} in complicating reconnaissance and lateral movement by continually altering IP addresses within a subnet, making it challenging for attackers to establish a static network view.
In future work, the potential of integrating this technique with Software Defined Networking (SDN) could offer more robust and flexible defense mechanisms. Additionally, addressing the limitations of MAC address fingerprinting and evaluating the impact on legitimate network users will further improve this Moving Target Defense strategy. Overall, the \texttt{ip-shuffle} script exemplifies proactive defense strategies that make it increasingly difficult for attackers to identify and exploit vulnerabilities.